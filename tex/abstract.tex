%!TEX program = xelatex

% -----------------------------------------------------------------------
% --------------------------------- PREAMBLE ----------------------------
% -----------------------------------------------------------------------

\documentclass[english, 10pt]{scrartcl}

\title{Dynamics of Community Formation in an Adaptive Voter Model}
\author{}
% \date{}
\date{\vspace{-3ex}}

% \usepackage{pc_writeup}
\usepackage{pc_math}
% \usepackage[comma,authoryear]{natbib} 
\usepackage[letterpaper, margin=1in]{geometry}
\usepackage{graphicx}

% -----------------------------------------------------------------------
% --------------------------------- BODY --------------------------------
% -----------------------------------------------------------------------

\begin{document}
\setkomafont{disposition}{\mdseries\rmfamily}

\maketitle

% Two sentences introducing the problem of dynamics in community structure. 
% Model description and overview of analytical properties. 
% Speculative connection to community structural concepts. 
	% SBM likelihood and modularity
	% Complex opinion structures
	% Parameter inference

Community structure is a fundamental object of study in a broad range of behavioral and biological networks, but canonical models are largely silent on the processes by which communities form and persist as networks evolve in time. Seeking a dynamical story of community structure, we study an adaptive voter model introduced by \cite{Kimura2008} and modified by \cite{Ji2013} in which node states and network topology coevolve. Agents in this model may update their opinions in response to those of their neighbors (``voting''); update their opinions randomly (``mutation''); or cut ties with neighbors with whom they disagree (``rewiring''). The model is parameterized by relative rates $\alpha$ of rewiring and $\lambda$ of mutation. Equilibria of this process display community structure in which both opinion labels and connection densities are emergent properties of the underlying dynamics. 

Following the approach of \cite{Durrett2012}, we develop two deterministic approximation schemes for the model's basic statistics. The first-order moment-closure method constructs a dynamical equation for the densities of different edge types, and allows a closed-form solution for the density of inter-opinion edges. The approximate master equation framework \cite{Gleeson2011}, on the other hand, leads to a considerable increase in accuracy at the cost of many more dynamical equations. These approximations lead to deterministic characterizations of both the system equilibria and the low-dimensional state-space manifolds that govern nonequilibrium behavior. 

We then consider in greater detail the community structures generated by this process. We describe the temporal behavior of standard metrics of community structure, including the stochastic blockmodel likelihood and the network modularity, using both simulations and deterministic approximations. Using structured opinion spaces, we then show that models of this class can generated a broad range of equilibrium community structures. Finally, we discuss approaches and challenges for learning system parameters from observed network snapshots. 

\begin{figure}
	\centering
	\includegraphics[width=.7\textwidth]{../../figs/PA.pdf}
	\caption{Equilibrium proportion of inter-opinion edges for a binary state mutating adaptive voter model. The proportion depends on both the rewiring rate $\alpha$ and mutation rate $\lambda$. Points give the average of five simulations from an agent-based simulation on $N = 1,000$ nodes with mean degree $c = 5$. In each simulation, a burn-in time of 20,000 steps was used, and then the proportion of inter-opinion edges was averaged over the following 20,000 steps. Solid lines give the analytic first-order moment-closure approximation.}
\end{figure}


\bibliography{/Users/phil/bibs/library.bib}{}
\bibliographystyle{unsrt}

\end{document}