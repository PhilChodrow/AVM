%!TEX program = xelatex
\documentclass{beamer}
\usepackage{pc_slides}
\usepackage{caption}
\usepackage{bbm}
\usepackage{subfig}
\usepackage{tikz}

\usetikzlibrary{arrows,shapes}

\usepackage[backend=bibtex, url=false, doi=false, isbn=false, eprint=false]{biblatex}
\bibliography{/Users/phil/bibs/library.bib}{}
\graphicspath{{../assets/}{../fig/}}

\title{Dynamics of Community Formation in an Adaptive Voter Model}
\date{\today}
\author{Philip S. Chodrow and Peter J. Mucha}
\institute{
	\vspace{6.4mm}
	$\vcenter{\hbox{\includegraphics[height=3cm]{PJM}}}$
	\hspace*{.5cm}
	$\vcenter{\hbox{\includegraphics[height=1cm]{mit_logo}}}$
	\hspace*{.5cm}
	$\vcenter{\hbox{\includegraphics[height=1.5cm]{UNC}}}$
	\hspace*{.5cm}
	% $\vcenter{\hbox{\includegraphics[height=1.5cm]{orc_logo}}}$
	% $\vcenter{\hbox{\includegraphics[height=1.5cm]{lids_logo}}}$
	% $\vcenter{\hbox{\includegraphics[height=1.5cm]{cee_logo}}}$
	$\vcenter{\hbox{\includegraphics[height=1.5cm]{nsf_logo}}}$
}

\newcommand\blfootnote[1]{%
  \begingroup
  \renewcommand\thefootnote{}\footnote{#1}%
  \addtocounter{footnote}{-1}%
  \endgroup
}

\metroset{block=fill}
\setbeamertemplate{frame footer}{$\quad$ Community Dynamics \alert| Chodrow \& Mucha}
\renewcommand{\footnotesize}{\tiny}

\newcommand\abs[1]{\left|#1\right|}
\newcommand\E[0]{\mathbb{E}}
\newcommand\prob[0]{\mathbb{P}}
\newcommand\R[0]{\mathbb{R}}
% -------------------------------------------------------------------------------------------------
% -------------------------------------------------------------------------------------------------
% -------------------------------------------------------------------------------------------------

\begin{document}
\maketitle

	%%%%%%%%%%%%%%%%%%%%%%%%%%%%%%%%%%%%%%%%%%%%%%%%%%%%%%%%%%%%%%%%%%%%%%%%%%%%%%%%%%%%%%%%%%%
		
		\begin{frame}\frametitle{Where does \alert{modularity} come from?}
		\begin{columns}
			\begin{column}{0.5\textwidth}
				\begin{figure}
			  		\centering
			  		\includegraphics[width=\textwidth]{community_network}
			  		\caption{\tiny{\fullcite{Karrer2011}}} 
			  	\end{figure}
			\end{column}
			\begin{column}{0.5\textwidth}
				How do \textbf{local dynamical mechanisms} generate and sustain \textbf{global modular structure} in networks? 
				\begin{itemize}
					\item Homophily
					\item Local influence
					\item Preferential growth
				\end{itemize}
			\end{column}
		\end{columns}
		\centering
		\end{frame}

	%%%%%%%%%%%%%%%%%%%%%%%%%%%%%%%%%%%%%%%%%%%%%%%%%%%%%%%%%%%%%%%%%%%%%%%%%%%%%%%%%%%%%%%%%%%
	%%%%%%%%%%%%%%%%%%%%%%%%%%%%%%%%%%%%%%%%%%%%%%%%%%%%%%%%%%%%%%%%%%%%%%%%%%%%%%%%%%%%%%%%%%%
		\setbeamertemplate{frame footer}{}
		\begin{frame}[standout]
			Can we formulate \alert{tractable}, \alert{dynamical} models of emergent modularity in networks?
		\end{frame}
		\setbeamertemplate{frame footer}{$\quad$ Community Dynamics \alert| Chodrow \& Mucha}
	%%%%%%%%%%%%%%%%%%%%%%%%%%%%%%%%%%%%%%%%%%%%%%%%%%%%%%%%%%%%%%%%%%%%%%%%%%%%%%%%%%%%%%%%%%%
	%%%%%%%%%%%%%%%%%%%%%%%%%%%%%%%%%%%%%%%%%%%%%%%%%%%%%%%%%%%%%%%%%%%%%%%%%%%%%%%%%%%%%%%%%%%
		
		\begin{frame}\frametitle{Noisy Adaptive Voter Models}
		  	Balance homophily and local influence via tunable parameters:  
		  	\begin{itemize}
		  		\item \textbf{Mutate}: With probability $\lambda$, change a node's opinion.\footfullcite{Ji2013}
		  		\item \textbf{Rewire-to-Random}: W. p. $(1-\lambda)\alpha$, rewire a discordant edge and attach it to a new random node.\footfullcite{Durrett2012}
		  		\item \textbf{Vote}: W. p. $(1-\alpha)(1-\lambda)$, a node adopts a neighbor's opinion. 
		  	\end{itemize}
		\end{frame}

	%%%%%%%%%%%%%%%%%%%%%%%%%%%%%%%%%%%%%%%%%%%%%%%%%%%%%%%%%%%%%%%%%%%%%%%%%%%%%%%%%%%%%%%%%%%
	%%%%%%%%%%%%%%%%%%%%%%%%%%%%%%%%%%%%%%%%%%%%%%%%%%%%%%%%%%%%%%%%%%%%%%%%%%%%%%%%%%%%%%%%%%%
		
		\begin{frame}\frametitle{Noisy Adaptive Voter Models}
			\begin{figure}
				\centering
					\includegraphics[width=.8\textwidth]{AVM_durrett}	
				\caption{\tiny{\fullcite{Durrett2012}}} 
			\end{figure}
		\end{frame}
	
	%%%%%%%%%%%%%%%%%%%%%%%%%%%%%%%%%%%%%%%%%%%%%%%%%%%%%%%%%%%%%%%%%%%%%%%%%%%%%%%%%%%%%%%%%%%
	%%%%%%%%%%%%%%%%%%%%%%%%%%%%%%%%%%%%%%%%%%%%%%%%%%%%%%%%%%%%%%%%%%%%%%%%%%%%%%%%%%%%%%%%%%%
		
		\begin{frame}\frametitle{Equilibrium Community Structure}
			\emph{\textbf{What's the expected modularity of this network at equilibrium?}}
			\begin{align*}
				q &= \E_\eta[Q] = \frac{1}{2m}\E_\eta\left[\left(\sum_{i,j \in \mathcal{N}^2} \mathbbm{1}[(i,j) \in \mathcal{E} ] - \frac{D_i D_j}{2m}\right)\mathbbm{1}[L_i = L_j]\right] \\ 
				&= \frac{1}{2} - \E_\eta\left[\frac{1}{m} \sum_{(i,j) \in \mathcal{E}} \mathbbm{1}[L_i = L_j]\right] \\ 
				&= \frac{1}{2} - \eta\left(L_i \neq L_j|(i,j) \in \mathcal{E}\right)\;.
			\end{align*}
		\end{frame}
	
	%%%%%%%%%%%%%%%%%%%%%%%%%%%%%%%%%%%%%%%%%%%%%%%%%%%%%%%%%%%%%%%%%%%%%%%%%%%%%%%%%%%%%%%%%%%
	%%%%%%%%%%%%%%%%%%%%%%%%%%%%%%%%%%%%%%%%%%%%%%%%%%%%%%%%%%%%%%%%%%%%%%%%%%%%%%%%%%%%%%%%%%%
		
		\begin{frame}\frametitle{Problem: These Systems are Hard!}
			\begin{enumerate}
				\item Highly nonlinear.
				\item Moment-closure methods (e.g. pair approximation) perform poorly. 
				\item Better methods scale badly with $c$ or number $k$ of opinions.
			\end{enumerate}
		  	\begin{figure}
		  		\centering
		  		\includegraphics[width=.9\textwidth]{bare_points}
		  	\end{figure}
		\end{frame}
	
	%%%%%%%%%%%%%%%%%%%%%%%%%%%%%%%%%%%%%%%%%%%%%%%%%%%%%%%%%%%%%%%%%%%%%%%%%%%%%%%%%%%%%%%%%%%
	%%%%%%%%%%%%%%%%%%%%%%%%%%%%%%%%%%%%%%%%%%%%%%%%%%%%%%%%%%%%%%%%%%%%%%%%%%%%%%%%%%%%%%%%%%%
		
		\begin{frame}\frametitle{Existing Approximation Methods}

			\begin{tabular}{l | c | c | c }
				Method & Accuracy & Generality &  Scaling \\ 
				\hline
				Pair Approximation\footnotemark & D & A &$O(k^2)$ \\ 
				AMEs\footnotemark[\value{footnote}]\footnotetext{\tiny{\fullcite{Durrett2012}}} 
				& B & A & $O(c^2 k^2 h^{-4})$ \\ 
				Active Motif\footfullcite{Demirel2012} & B & B & $O(c^2k^2\log_2\frac{1}{\sqrt{\epsilon}})$\\ 
				\textbf{Markov (Today)} & \textbf{A} & \textbf{B} & $O(k^2\log_2\frac{1}{\sqrt{\epsilon}})$
			\end{tabular}

			% \begin{itemize}
			% 	\item \textbf{Pair Approximation:} Scalable, inaccurate. \cite{Demirel2012,Durrett2012}
			% 	\item \textbf{Approximate Master Equations:} Somewhat accurate, not scalable with mean degree, not scalable with number of opinion types. \cite{Durrett2012}
			% 	\item \textbf{Active Motif Approximation:} Somewhat accurate, scalable with degree, not scalable with number of opinion types. \cite{Demirel2012}
			% \end{itemize}

		\end{frame}
	
	%%%%%%%%%%%%%%%%%%%%%%%%%%%%%%%%%%%%%%%%%%%%%%%%%%%%%%%%%%%%%%%%%%%%%%%%%%%%%%%%%%%%%%%%%%%
	%%%%%%%%%%%%%%%%%%%%%%%%%%%%%%%%%%%%%%%%%%%%%%%%%%%%%%%%%%%%%%%%%%%%%%%%%%%%%%%%%%%%%%%%%%%
		\setbeamertemplate{frame footer}{}
		\begin{frame}[standout]
			Can we formulate \alert{tractable}, \alert{dynamical} models of emergent modularity in networks?
		\end{frame}
		\setbeamertemplate{frame footer}{$\quad$ Community Dynamics \alert| Chodrow \& Mucha}
	%%%%%%%%%%%%%%%%%%%%%%%%%%%%%%%%%%%%%%%%%%%%%%%%%%%%%%%%%%%%%%%%%%%%%%%%%%%%%%%%%%%%%%%%%%%
	%%%%%%%%%%%%%%%%%%%%%%%%%%%%%%%%%%%%%%%%%%%%%%%%%%%%%%%%%%%%%%%%%%%%%%%%%%%%%%%%%%%%%%%%%%%
		
		\begin{frame}\frametitle{A Markov Equilibrium Approximation}
			At equilibrium, processes balance: 
			\begin{align*}
				0 &= \underbrace{2\lambda c\E_\eta[Q]}_{\text{Mutation}}  - \overbrace{(1-\lambda)\frac{\alpha}{2}}^{\text{Rewiring}} + \underbrace{(1-\lambda)(1-\alpha)\mathbb{E}_\eta[V(\mathcal{G})]}_{\text{Voting}}\;, 
			\end{align*}
			where $V$ is an unknown term reflecting the expected impact of a voting event on $Q$. 

			\textbf{Strategy:} Assume that $V(\mathcal{G}) \approx V(Q(\mathcal{G}))$ and estimate it; then solve for $\E_\eta[Q]$.
		\end{frame}
	
	%%%%%%%%%%%%%%%%%%%%%%%%%%%%%%%%%%%%%%%%%%%%%%%%%%%%%%%%%%%%%%%%%%%%%%%%%%%%%%%%%%%%%%%%%%%
	%%%%%%%%%%%%%%%%%%%%%%%%%%%%%%%%%%%%%%%%%%%%%%%%%%%%%%%%%%%%%%%%%%%%%%%%%%%%%%%%%%%%%%%%%%%
		
		\begin{frame}\frametitle{Approximating the Voter Term}
		  	\begin{columns}
		  		\begin{column}{0.4\textwidth}
					\tikzstyle{vertex}=[circle, fill=blue!30, minimum size=20pt, line width=0mm, inner sep=0pt]
					\tikzstyle{selected vertex} = [vertex, fill=orange!50, line width=0mm]
					\tikzstyle{discordant edge} = [draw,line width=1mm, -]
					\tikzstyle{concordant edge} = [draw,line width=1mm, -,gray]
					\tikzstyle{weight} = [font=\small]

					\begin{figure}
						\centering     
							\begin{tikzpicture}[scale=1, auto,swap]
							    
							    \node[selected vertex] (v) at (0, 0) {$v$};
							    \node[selected vertex] (v_0) at (0, -2) {$v_0$};
							    \foreach \pos/\name in {{(-2,0)/v_1}, {(0,2)/v_2}, {(2,0)/v_3}}
							        \node[vertex] (\name) at \pos {$\name$};
							    % Connect vertices with edges and draw weights
							    \foreach \target in {v_1, v_2, v_3}
							    	\draw[discordant edge] (v) --  (\target);
							    \draw[concordant edge] (v) -- (v_0);
							\end{tikzpicture}
						\label{fig:examples}
					\end{figure}
				\end{column}
				\begin{column}{0.6\textwidth}
					\textbf{Scheme:} 
					\begin{enumerate}
						\item Assume that a node $v$ has just changed its vote. 
						\item Track each discordant edge until resolved.
						\item Assume mean-field, separation of timescales, regularity. 
						\item Approximate $\E_\eta[V(\mathcal{G})]$ as the average change in modularity due to voting events along the way. 
					\end{enumerate}
				\end{column}
			\end{columns}
		\end{frame}
	
	%%%%%%%%%%%%%%%%%%%%%%%%%%%%%%%%%%%%%%%%%%%%%%%%%%%%%%%%%%%%%%%%%%%%%%%%%%%%%%%%%%%%%%%%%%%
	%%%%%%%%%%%%%%%%%%%%%%%%%%%%%%%%%%%%%%%%%%%%%%%%%%%%%%%%%%%%%%%%%%%%%%%%%%%%%%%%%%%%%%%%%%%
		
		\begin{frame}\frametitle{Subcritical Approximation}
			In subcritical regime ($q = 1/2$): 
			\begin{align*}
				\hat{V}\left(\frac{1}{2}\right) &=  \frac{\mathbb{P}(v \text{ votes})\E[\text{change when } v \text{ votes}] + \E[\text{other votes}](c-1)}{\mathbb{P}(v \text{ votes}) + \E[\text{other votes}]} \\ 
				% &= \frac{-\rho\sigma(1+\eta) + \rho(\eta - 1) + \frac{c}{1-\beta^c}(\eta \beta^c - 1)}{2+\rho\sigma(1+\eta)}\;. \\
				&= \frac{\rho(\eta - 1 - \sigma(1+\eta)) - c\frac{1 - \eta \beta^c}{1-\beta^c}}{2+\rho\sigma(1+\eta)}
			\end{align*}
			$\rho$, $\sigma$, $\eta$, $\beta$ are all functions of the rewiring rate $\alpha$ and mean opinion $u = \prob(L = 1)$. 
		\end{frame}
	
	%%%%%%%%%%%%%%%%%%%%%%%%%%%%%%%%%%%%%%%%%%%%%%%%%%%%%%%%%%%%%%%%%%%%%%%%%%%%%%%%%%%%%%%%%%%
	%%%%%%%%%%%%%%%%%%%%%%%%%%%%%%%%%%%%%%%%%%%%%%%%%%%%%%%%%%%%%%%%%%%%%%%%%%%%%%%%%%%%%%%%%%%
		
		\begin{frame}\frametitle{Phase Transition}
		  		{ \centering
		  			\includegraphics[width=.6\textwidth]{transition_joy_random}\par
		  		}
		  		Grey densities give $\frac{1}{2} - q$\; for $\lambda = 2^{-10}$.
		  		
		  		Red curve obtained by numerically solving for $\alpha^*$:
		  		\begin{align*}
		  			 \lambda c -(1-\lambda)\frac{\alpha}{2} + (1-\lambda)(1-\alpha)\hat{V}\left(\frac{1}{2}\right) = 0\;.
		  		\end{align*}
		  	
		\end{frame}
	
	%%%%%%%%%%%%%%%%%%%%%%%%%%%%%%%%%%%%%%%%%%%%%%%%%%%%%%%%%%%%%%%%%%%%%%%%%%%%%%%%%%%%%%%%%%%
	%%%%%%%%%%%%%%%%%%%%%%%%%%%%%%%%%%%%%%%%%%%%%%%%%%%%%%%%%%%%%%%%%%%%%%%%%%%%%%%%%%%%%%%%%%%
		
		\begin{frame}\frametitle{Supercritical Behavior}
			
			How do we move past $\alpha^*$? Linear approximation:  
			\begin{align*}
				\hat{V}(q) \approx \frac{V(q_0) - V\left(\frac{1}{2}\right)}{2q_0 -1}(2q - 1) + V\left(\frac{1}{2}\right)
			\end{align*}
			where $q_0$, $V(q_0)$ are computed from known results about $\alpha = 0$.		
		
			\begin{figure}
				% \centering
				\includegraphics[width=.85\textwidth]{V_term_empirical_sub.pdf}
				\caption{} 
			\end{figure}

			\blfootnote{\fullcite{Allen2012}}
		\end{frame}
	
	%%%%%%%%%%%%%%%%%%%%%%%%%%%%%%%%%%%%%%%%%%%%%%%%%%%%%%%%%%%%%%%%%%%%%%%%%%%%%%%%%%%%%%%%%%%
	%%%%%%%%%%%%%%%%%%%%%%%%%%%%%%%%%%%%%%%%%%%%%%%%%%%%%%%%%%%%%%%%%%%%%%%%%%%%%%%%%%%%%%%%%%%
		
		\begin{frame}\frametitle{Supercritical Behavior}
		  	\begin{figure}
		  		\centering
		  		\includegraphics[width=\textwidth]{PA_comparison}
		  		\caption{$\lambda = 2^{-10}$} 
		  	\end{figure}
		\end{frame}	
	%%%%%%%%%%%%%%%%%%%%%%%%%%%%%%%%%%%%%%%%%%%%%%%%%%%%%%%%%%%%%%%%%%%%%%%%%%%%%%%%%%%%%%%%%%%
	%%%%%%%%%%%%%%%%%%%%%%%%%%%%%%%%%%%%%%%%%%%%%%%%%%%%%%%%%%%%%%%%%%%%%%%%%%%%%%%%%%%%%%%%%%%
		
		\begin{frame}\frametitle{Generalizations}
			\begin{itemize}
				\item Multiple opinion types.\footfullcite{Shi2013}
			  	\item Structured opinion spaces, e.g. $L \in \{-2, -1, 0, 1, 2\}$.
			  	\item Variable mutation and rewiring rates.
			\end{itemize}
		\end{frame}
	
	%%%%%%%%%%%%%%%%%%%%%%%%%%%%%%%%%%%%%%%%%%%%%%%%%%%%%%%%%%%%%%%%%%%%%%%%%%%%%%%%%%%%%%%%%%%
	%%%%%%%%%%%%%%%%%%%%%%%%%%%%%%%%%%%%%%%%%%%%%%%%%%%%%%%%%%%%%%%%%%%%%%%%%%%%%%%%%%%%%%%%%%%
		
		\begin{frame}\frametitle{Wrapping Up}
			\begin{enumerate}
			  	\item Adaptive Voter Models model the emergence of global modular structure from local dynamics.
			  	\item Our Markovian approximation is best-in-class on both accuracy and computational efficiency for the random-rewiring AVM. 
			  	\item Quick computation opens the door to tractable generalizations: multiple opinions; structured opinion spaces;  asymmetric mutation. 
		  	\end{enumerate}  
		\end{frame}
	
	%%%%%%%%%%%%%%%%%%%%%%%%%%%%%%%%%%%%%%%%%%%%%%%%%%%%%%%%%%%%%%%%%%%%%%%%%%%%%%%%%%%%%%%%%%%
	\setbeamertemplate{frame footer}{}
		\begin{frame}[standout]
			Thanks! Questions? 
		\end{frame}
		\setbeamertemplate{frame footer}{$\quad$ Community Dynamics \alert| Chodrow \& Mucha}
	%%%%%%%%%%%%%%%%%%%%%%%%%%%%%%%%%%%%%%%%%%%%%%%%%%%%%%%%%%%%%%%%%%%%%%%%%%%%%%%%%%%%%%%%%%%
\section{Additional Material}
	%%%%%%%%%%%%%%%%%%%%%%%%%%%%%%%%%%%%%%%%%%%%%%%%%%%%%%%%%%%%%%%%%%%%%%%%%%%%%%%%%%%%%%%%%%%
	
		\begin{frame}[standout]
		References
		  
		\end{frame}

	%%%%%%%%%%%%%%%%%%%%%%%%%%%%%%%%%%%%%%%%%%%%%%%%%%%%%%%%%%%%%%%%%%%%%%%%%%%%%%%%%%%%%%%%%%%
	%%%%%%%%%%%%%%%%%%%%%%%%%%%%%%%%%%%%%%%%%%%%%%%%%%%%%%%%%%%%%%%%%%%%%%%%%%%%%%%%%%%%%%%%%%%
		
		\begin{frame}\frametitle{Misc math}
		  \begin{align*}	
		  		\prob(v \text{ votes}) &= 1-\beta^c \\ 
				\E[\text{change when } v \text{ votes}] &= \frac{c}{1-\beta^{c}} - \frac{\beta}{1-\beta} \\ 
				\E[\text{other votes}] &= (1-\beta^c)\left(1 + \sigma (1+\eta) \frac{\beta}{1-\beta}\right) \\ 
				w(q) &= 2q - 1
		  	\end{align*}
		  	$\beta$, $\eta$, and $\sigma$ are coefficients depending on $\alpha$ and the mean opinion $\prob(L = 1)$. 
		\end{frame}
	
	%%%%%%%%%%%%%%%%%%%%%%%%%%%%%%%%%%%%%%%%%%%%%%%%%%%%%%%%%%%%%%%%%%%%%%%%%%%%%%%%%%%%%%%%%%%
\end{document}
