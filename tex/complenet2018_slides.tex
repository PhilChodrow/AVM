%!TEX program = xelatex
\documentclass{beamer}
\usepackage{pc_slides}
\usepackage{caption}


\graphicspath{{../assets/}{../fig/}}

\title{Dynamics of Community Formation in an Adaptive Voter Model}
\date{\today}
\author{Philip S. Chodrow and Peter J. Mucha}
\institute{
	\vspace{6.4mm}
	$\vcenter{\hbox{\includegraphics[height=1cm]{mit_logo}}}$
	\hspace*{.2in}
	% $\vcenter{\hbox{\includegraphics[height=1.5cm]{orc_logo}}}$
	% $\vcenter{\hbox{\includegraphics[height=1.5cm]{lids_logo}}}$
	% $\vcenter{\hbox{\includegraphics[height=1.5cm]{cee_logo}}}$
	$\vcenter{\hbox{\includegraphics[height=1.5cm]{nsf_logo}}}$
}

\metroset{block=fill}
\setbeamertemplate{frame footer}{$\quad$ Community Dynamics \alert| Chodrow \& Mucha}

% -------------------------------------------------------------------------------------------------
% -------------------------------------------------------------------------------------------------
% -------------------------------------------------------------------------------------------------

\begin{document}
\maketitle

\section{Introduction}
	%%%%%%%%%%%%%%%%%%%%%%%%%%%%%%%%%%%%%%%%%%%%%%%%%%%%%%%%%%%%%%%%%%%%%%%%%%%%%%%%%%%%%%%%%%%
		
		\begin{frame}\frametitle{Community Structure}
		  
		\end{frame}

	%%%%%%%%%%%%%%%%%%%%%%%%%%%%%%%%%%%%%%%%%%%%%%%%%%%%%%%%%%%%%%%%%%%%%%%%%%%%%%%%%%%%%%%%%%%
	%%%%%%%%%%%%%%%%%%%%%%%%%%%%%%%%%%%%%%%%%%%%%%%%%%%%%%%%%%%%%%%%%%%%%%%%%%%%%%%%%%%%%%%%%%%
		
		\begin{frame}\frametitle{A Familiar Community Model}
		  
			But: how do networks \emph{get this way?} 
		\end{frame}
	
	%%%%%%%%%%%%%%%%%%%%%%%%%%%%%%%%%%%%%%%%%%%%%%%%%%%%%%%%%%%%%%%%%%%%%%%%%%%%%%%%%%%%%%%%%%%
	%%%%%%%%%%%%%%%%%%%%%%%%%%%%%%%%%%%%%%%%%%%%%%%%%%%%%%%%%%%%%%%%%%%%%%%%%%%%%%%%%%%%%%%%%%%
		
		\begin{frame}[standout]
			So, can we formulate \alert{tractable}, \alert{dynamical} models of community structure in networks?
		\end{frame}
	
	%%%%%%%%%%%%%%%%%%%%%%%%%%%%%%%%%%%%%%%%%%%%%%%%%%%%%%%%%%%%%%%%%%%%%%%%%%%%%%%%%%%%%%%%%%%
	%%%%%%%%%%%%%%%%%%%%%%%%%%%%%%%%%%%%%%%%%%%%%%%%%%%%%%%%%%%%%%%%%%%%%%%%%%%%%%%%%%%%%%%%%%%
		
		\begin{frame}\frametitle{Noisy Adaptive Voter Models}
		  	Network $G$, nodes with opinions $\{\ell_i\}$. Each time-step:
		  	\begin{itemize}
		  		\item \textbf{Mutation}: With probability $\lambda$, flip a node's opinion.
		  		\item \textbf{Rewiring}: With probability $(1-\lambda)\alpha$, move a \textbf{discordant} edge between two nodes that disagree and attach it to a new node selected uniformly at random. 
		  		\item \textbf{Vote}: With probability $(1-\alpha)(1-\lambda)$, a node attached to a discordant discordant edge changes its opinion. 
		  	\end{itemize}
		  	\cite{Durrett2012}, more citations here
		  	\textbf{Note:} When $\lambda > 0$, system is ergodic. 
		\end{frame}

	%%%%%%%%%%%%%%%%%%%%%%%%%%%%%%%%%%%%%%%%%%%%%%%%%%%%%%%%%%%%%%%%%%%%%%%%%%%%%%%%%%%%%%%%%%%
	%%%%%%%%%%%%%%%%%%%%%%%%%%%%%%%%%%%%%%%%%%%%%%%%%%%%%%%%%%%%%%%%%%%%%%%%%%%%%%%%%%%%%%%%%%%
		
		\begin{frame}\frametitle{Simple Analytic Question}
			Let $\eta$ be the equilibrium measure, and define 
			\begin{align*}
				x = \mathbb{E}_\eta[X] = \mathbb{E}_\eta\left[\frac{1}{m}\sum_{(i,j) \in \mathcal{E}} \delta (\ell_i, \ell_j)\right],
			\end{align*}
			the \textbf{expected proportion of discordant edges}. 

			How can we approximate $x$ in terms of $\lambda$, $\alpha$, and $c$?
		\end{frame}
	
	%%%%%%%%%%%%%%%%%%%%%%%%%%%%%%%%%%%%%%%%%%%%%%%%%%%%%%%%%%%%%%%%%%%%%%%%%%%%%%%%%%%%%%%%%%%
	%%%%%%%%%%%%%%%%%%%%%%%%%%%%%%%%%%%%%%%%%%%%%%%%%%%%%%%%%%%%%%%%%%%%%%%%%%%%%%%%%%%%%%%%%%%
		
		\begin{frame}\frametitle{Existing Analytic Methods}

			\begin{itemize}
				\item \textbf{Pair Approximation:} Scalable, inaccurate. \cite{Demirel2012,Durrett2012}
				\item \textbf{Approximate Master Equations:} Somewhat accurate, not scalable with mean degree, not scalable with number of opinion types. \cite{Durrett2012}
				\item \textbf{Active Motif Approximation:} Somewhat accurate, scalable with degree, not scalable with number of opinion types. \cite{Demirel2012}
			\end{itemize}

		\end{frame}
	
	%%%%%%%%%%%%%%%%%%%%%%%%%%%%%%%%%%%%%%%%%%%%%%%%%%%%%%%%%%%%%%%%%%%%%%%%%%%%%%%%%%%%%%%%%%%
	%%%%%%%%%%%%%%%%%%%%%%%%%%%%%%%%%%%%%%%%%%%%%%%%%%%%%%%%%%%%%%%%%%%%%%%%%%%%%%%%%%%%%%%%%%%
		
		\begin{frame}\frametitle{An Equilibrium Approximation}
			For any measure $\nu$, we have 
			\begin{align*}
				\mathbb{E}_\nu[X_{t+1} - X_{t}] &= \lambda c(1-2\mathbb{E}_\nu[X_t])  \tag{mutate} \\ 
											&\quad - (1-\lambda)\frac{\alpha}{2} \tag{rewire} \\ 
											&\quad + (1-\lambda)(1-\alpha)\mathbb{E}_\nu[V(X_t)]\;, \tag{vote}
			\end{align*}
			where $V$ is an unknown term reflecting the expected impact of a vote on $X_t$. 

			\textbf{Strategy:} Approximate $V$ at equilibrium, take $\nu = \eta$, and solve for $\mathbb{E}_\eta[X_t]$.

		\end{frame}
	
	%%%%%%%%%%%%%%%%%%%%%%%%%%%%%%%%%%%%%%%%%%%%%%%%%%%%%%%%%%%%%%%%%%%%%%%%%%%%%%%%%%%%%%%%%%%
	%%%%%%%%%%%%%%%%%%%%%%%%%%%%%%%%%%%%%%%%%%%%%%%%%%%%%%%%%%%%%%%%%%%%%%%%%%%%%%%%%%%%%%%%%%%
		
		\begin{frame}\frametitle{Approximation Scheme}
		  	Scheme: consider a simple scenario, ignore interactions, and then average the voter term within this scenario, resulting in an analytically tractable approximation. 
		\end{frame}
	
	%%%%%%%%%%%%%%%%%%%%%%%%%%%%%%%%%%%%%%%%%%%%%%%%%%%%%%%%%%%%%%%%%%%%%%%%%%%%%%%%%%%%%%%%%%%
	%%%%%%%%%%%%%%%%%%%%%%%%%%%%%%%%%%%%%%%%%%%%%%%%%%%%%%%%%%%%%%%%%%%%%%%%%%%%%%%%%%%%%%%%%%%
		
		\begin{frame}\frametitle{Phase Transition}
		  	\begin{figure}
		  		\centering
		  		\includegraphics[width=\textwidth]{transition_joy.pdf}
		  		\caption{Caption!} \label{fig:}
		  	\end{figure}
		\end{frame}
	
	%%%%%%%%%%%%%%%%%%%%%%%%%%%%%%%%%%%%%%%%%%%%%%%%%%%%%%%%%%%%%%%%%%%%%%%%%%%%%%%%%%%%%%%%%%%
	%%%%%%%%%%%%%%%%%%%%%%%%%%%%%%%%%%%%%%%%%%%%%%%%%%%%%%%%%%%%%%%%%%%%%%%%%%%%%%%%%%%%%%%%%%%
		
		\begin{frame}\frametitle{Equilibrium Behavior}
		  	\begin{figure}
		  		\centering
		  		\includegraphics[width=\textwidth]{full_approx_lambda_0.pdf}
		  		\caption{} \label{fig:}
		  	\end{figure}
		\end{frame}	
	%%%%%%%%%%%%%%%%%%%%%%%%%%%%%%%%%%%%%%%%%%%%%%%%%%%%%%%%%%%%%%%%%%%%%%%%%%%%%%%%%%%%%%%%%%%
	%%%%%%%%%%%%%%%%%%%%%%%%%%%%%%%%%%%%%%%%%%%%%%%%%%%%%%%%%%%%%%%%%%%%%%%%%%%%%%%%%%%%%%%%%%%
		
		\begin{frame}\frametitle{Scalability}
		  	\begin{itemize}
		  		\item Pair Approximation: $O(k^2)$.
		  		\item Approximate Master Equations: $O(c^2 k^2)$.
		  		\item Active Motif Approximations: $O(c^3 k)$.
		  		\item Present Method: $O(k^2)$. 
		  	\end{itemize}
		\end{frame}
	
	%%%%%%%%%%%%%%%%%%%%%%%%%%%%%%%%%%%%%%%%%%%%%%%%%%%%%%%%%%%%%%%%%%%%%%%%%%%%%%%%%%%%%%%%%%%
\section{New Analytic Results}
	%%%%%%%%%%%%%%%%%%%%%%%%%%%%%%%%%%%%%%%%%%%%%%%%%%%%%%%%%%%%%%%%%%%%%%%%%%%%%%%%%%%%%%%%%%%
		
		\begin{frame}\frametitle{}
		  
		\end{frame}

	%%%%%%%%%%%%%%%%%%%%%%%%%%%%%%%%%%%%%%%%%%%%%%%%%%%%%%%%%%%%%%%%%%%%%%%%%%%%%%%%%%%%%%%%%%%
\section{Community Measures}
	%%%%%%%%%%%%%%%%%%%%%%%%%%%%%%%%%%%%%%%%%%%%%%%%%%%%%%%%%%%%%%%%%%%%%%%%%%%%%%%%%%%%%%%%%%%
		
		\begin{frame}\frametitle{}
		  	
		\end{frame}

	%%%%%%%%%%%%%%%%%%%%%%%%%%%%%%%%%%%%%%%%%%%%%%%%%%%%%%%%%%%%%%%%%%%%%%%%%%%%%%%%%%%%%%%%%%%
\section{Conjectures}
	%%%%%%%%%%%%%%%%%%%%%%%%%%%%%%%%%%%%%%%%%%%%%%%%%%%%%%%%%%%%%%%%%%%%%%%%%%%%%%%%%%%%%%%%%%%
		
		\begin{frame}\frametitle{}
		  
		\end{frame}

	%%%%%%%%%%%%%%%%%%%%%%%%%%%%%%%%%%%%%%%%%%%%%%%%%%%%%%%%%%%%%%%%%%%%%%%%%%%%%%%%%%%%%%%%%%%
\section{Wrapping Up}
	%%%%%%%%%%%%%%%%%%%%%%%%%%%%%%%%%%%%%%%%%%%%%%%%%%%%%%%%%%%%%%%%%%%%%%%%%%%%%%%%%%%%%%%%%%%
		
		\begin{frame}\frametitle{}
		  
		\end{frame}

	%%%%%%%%%%%%%%%%%%%%%%%%%%%%%%%%%%%%%%%%%%%%%%%%%%%%%%%%%%%%%%%%%%%%%%%%%%%%%%%%%%%%%%%%%%%
	%%%%%%%%%%%%%%%%%%%%%%%%%%%%%%%%%%%%%%%%%%%%%%%%%%%%%%%%%%%%%%%%%%%%%%%%%%%%%%%%%%%%%%%%%%%
	
		\begin{frame}[standout]
		References
		  
		\end{frame}

	%%%%%%%%%%%%%%%%%%%%%%%%%%%%%%%%%%%%%%%%%%%%%%%%%%%%%%%%%%%%%%%%%%%%%%%%%%%%%%%%%%%%%%%%%%%
	%%%%%%%%%%%%%%%%%%%%%%%%%%%%%%%%%%%%%%%%%%%%%%%%%%%%%%%%%%%%%%%%%%%%%%%%%%%%%%%%%%%%%%%%%%%
	
		\begin{frame}[allowframebreaks]\frametitle{References}
			\bibliographystyle{apalike}
		    \bibliography{/Users/phil/bibs/library.bib}{}
		\end{frame}

	%%%%%%%%%%%%%%%%%%%%%%%%%%%%%%%%%%%%%%%%%%%%%%%%%%%%%%%%%%%%%%%%%%%%%%%%%%%%%%%%%%%%%%%%%%%

\end{document}
